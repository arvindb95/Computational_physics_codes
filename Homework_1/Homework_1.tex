\documentclass[a4paper,10pt]{article}
\usepackage[utf8]{inputenc}
\usepackage{amsmath}
\usepackage{xfrac}
\usepackage[a4paper, total={7.5in, 10in}]{geometry}
\usepackage{longtable}
\usepackage[table]{xcolor}
\usepackage{listings}
\lstset{
breaklines=true
}
%http://tex.stackexchange.com/questions/116534/lstlisting-line-wrapping
\usepackage{hyperref}
\hypersetup{
    colorlinks=true,
    linkcolor=blue,
    citecolor=blue
}
\usepackage{color}
 
\definecolor{codegreen}{rgb}{0,0.6,0}
\definecolor{codegray}{rgb}{0.5,0.5,0.5}
\definecolor{codepurple}{rgb}{0.58,0,0.82}
\definecolor{backcolour}{rgb}{0.95,0.95,0.92}
 
\lstdefinestyle{mystyle}{
    backgroundcolor=\color{backcolour},   
    commentstyle=\color{codegreen},
    keywordstyle=\color{magenta},
    numberstyle=\tiny\color{codegray},
    stringstyle=\color{codepurple},
    basicstyle=\footnotesize,
    breakatwhitespace=false,         
    breaklines=true,                 
    captionpos=b,                    
    keepspaces=true,                 
    numbers=left,                    
    numbersep=5pt,                  
    showspaces=false,                
    showstringspaces=false,
    showtabs=false,                  
    tabsize=2
}
 
\lstset{style=mystyle}

%opening

\title{\textbf{Homework 1 : Numerical Integration}}
\author{Arvind Balasubramanian}

\begin{document}
\maketitle
\section*{Removing the singularities in the function}

The integral to be calculated is :
\begin{equation*}
 \boxed{I = \int_{0}^{1} t^{\sfrac{-2}{3}} (1-t)^{\sfrac{-1}{3}} dt}
\end{equation*}
As we can see there are singularities at 0 and 1. First, let us break this integral into two parts as follows :

\begin{equation*}
 I = \int_{0}^{\sfrac{1}{2}} t^{\sfrac{-2}{3}} (1-t)^{\sfrac{-1}{3}} dt + \int_{\sfrac{1}{2}}^{1} t^{\sfrac{-2}{3}} (1-t)^{\sfrac{-1}{3}} dt
\end{equation*}

Consider only the first integral on the right in the above equation. Let us make a substitution $ u = t^{\sfrac{1}{3}}$. So, we get :
\begin{equation*}
 \begin{split}
  u = t^{\sfrac{1}{3}} & \implies du = \frac{1}{3} t^{\sfrac{-2}{3}} dt \\
  \\
  \text{At } t = 0 \text{ ; } u = 0 & \text{ and at } t = \frac{1}{2} \text{ ; } u = \Big(\frac{1}{2}\Big)^{\sfrac{1}{3}}\\
  \\
  \end{split}  
\end{equation*}

Now, the singularity at 0 has been taken care of and the integral becomes :
\begin{equation*}
 \int_{0}^{\sfrac{1}{2}} t^{\sfrac{-2}{3}} (1-t)^{\sfrac{-1}{3}} dt = 3 \int_{0}^{0.5^{\sfrac{1}{3}}} (1-u^{3})^{\sfrac{-1}{3}} du
\end{equation*}

Similarly, consider the second integral. Let us make a substitution $ s = (1-t)^{\sfrac{2}{3}}$. So, we get :
\begin{equation*}
 \begin{split}
  s = (1-t)^{\sfrac{2}{3}} & \implies ds = \frac{-2}{3} (1-t)^{\sfrac{-1}{3}} dt \\
  \\
  \text{At } t = \frac{1}{2} \text{ ; } s & = \Big(\frac{1}{2}\Big)^{\sfrac{2}{3}} \text{ and at } t = 1 \text{ ; } s = 0\\
  \\
  \end{split}  
\end{equation*}

Now, the singularity at 1 has been taken care of and the integral becomes :
\begin{equation*}
 \int_{\sfrac{1}{2}}^{1} t^{\sfrac{-2}{3}} (1-t)^{\sfrac{-1}{3}} dt = \frac{3}{2}\int_{0}^{0.5^{\sfrac{2}{3}}} (1 - s^{\sfrac{3}{2}})^{\sfrac{-2}{3}} ds
\end{equation*}

Therefore, finally, the integral $I$ becomes :
\begin{equation*}
  \boxed{I = 3 \int_{0}^{0.5^{\sfrac{1}{3}}} (1-u^{3})^{\sfrac{-1}{3}} du + \frac{3}{2}\int_{0}^{0.5^{\sfrac{2}{3}}} (1 - s^{\sfrac{3}{2}})^{\sfrac{-2}{3}} ds}
\end{equation*}

\newpage

\section*{Python code and analysis}
\begin{lstlisting}[language=python]
#!/usr/bin/python

import numpy as np

def trapezoidal(function,a,b,stepsize):
    """
    Function that takes a function, lower limit a, upper limit b and stepsize and calculates the integral using trapezoidal method.

    """
    interval = np.arange(a,b+stepsize,stepsize) ## array of the interval points
    integral = 0 ## variable to store the sum
    for i in range(1,len(interval)-1,2):
        integral += (stepsize/2)*(function(interval[i-1]) + 2*function(interval[i]) + function(interval[i+1]))
    return integral

def simpsons(function,a,b,stepsize):
    """
    Function that takes a function, lower limit a, upper limit b and stepsize and calculates the integral using simpsons method.
    """
    interval = np.arange(a,b+stepsize,stepsize) ## array of interrval points
    integral = 0 ## variable to store the sum
    for i in range(1,len(interval)-1,2):
        integral += (stepsize/3)*(function(interval[i-1]) + 4*function(interval[i]) + function(interval[i+1]))
    return integral                           

######### Define your function and limits of integration ############

def func_l(u):
    return 3*(1 - (u)**3)**(-1/3)

def func_r(s):
    return (3/2)*(1 - (s)**(3/2))**(-2/3)

a_l = 0 
b_l = (0.5)**(1/3)
no_of_bins = 10**3
stepsize_l = (b_l - a_l)/no_of_bins

a_r = 0
b_r = (0.5)**(2/3)
stepsize_r = (b_r - a_r)/no_of_bins

# The final answer in trapezoidal and simpsons methods respectively are
t_inte = trapezoidal(func_l,a_l,b_l,stepsize_l) + trapezoidal(func_r,a_r,b_r,stepsize_r)
s_inte = simpsons(func_l,a_l,b_l,stepsize_l) + simpsons(func_r,a_r,b_r,stepsize_r)

\end{lstlisting}

The outputs are shown below and the values that already converge are shaded

\begin{center}
\begin{longtable}{|c|c|c|c|c|c|c|}
\hline
No. of bins & Step size & Step size & Trapezoidal & $\frac{Trapezoidal}{Theoretical}$ & Simpsons & $\frac{Simpsons}{Theoretical}$ \\
 & (left) & (right) & & & & \\ 
\hline
\hline
10 & 0.079370 & 0.062996 & 3.631300 & 1.001020 & 3.627696 & 1.000026 \\
\hline
100 & 0.007937 & 0.006300 & 3.627636 & 1.000010 & \cellcolor{codegreen}3.627599 & 1.000000 \\
\hline
1000 & 0.000794 & 0.000630 & \cellcolor{codegreen}3.627599 & 1.000000 & \cellcolor{codegreen}3.627599 & 1.000000 \\
\hline
10000 & 7.93700e-05 & 6.29960e-05 & \cellcolor{codegreen}3.627599 & 1.000000 & \cellcolor{codegreen}3.627599 & 1.000000 \\
\hline
\hline
\end{longtable}
\end{center}


\newpage

\section*{C++ code and analysis}
\begin{lstlisting}[language=c++]
#include <iostream>
#include <math.h>
using namespace std;

double f_l(double u){
        double e = -1.0/3.0;
        return 3.0*pow((1 - pow(u,3.0)),e);
}

double f_r(double s){
        double g = 3.0/2.0;
        double h = -2.0/3.0;
        return g*pow((1 - pow(s,g)),h);

}

int main(){
        double a_l = 0.0;
        double b_l = pow(0.5,1.0/3.0);
        double number_of_steps = pow(10,4);
        double stepsize_l = (b_l - a_l)/number_of_steps;

        double trap_l = 0;
        double simp_l = 0;
        for (double i = a_l + stepsize_l; i < b_l; i += 2.0*stepsize_l){
                trap_l += (stepsize_l/2.0)*(f_l(i-stepsize_l) + 2.0*f_l(i) + f_l(i+stepsize_l));
                simp_l += (stepsize_l/3.0)*(f_l(i-stepsize_l) + 4.0*f_l(i) + f_l(i+stepsize_l));
        }


        double a_r = 0.0;
        double b_r = pow(0.5,2.0/3.0);
        double stepsize_r = (b_r - a_r)/number_of_steps;

        double trap_r = 0;
        double simp_r = 0;
        for (double i = a_r + stepsize_r; i < b_r; i += 2.0*stepsize_r){
                trap_r += (stepsize_r/2.0)*(f_r(i-stepsize_r) + 2.0*f_r(i) + f_r(i+stepsize_r));
                simp_r += (stepsize_r/3.0)*(f_r(i-stepsize_r) + 4.0*f_r(i) + f_r(i+stepsize_r));
        }       

        double final_trap = trap_l + trap_r;
        double final_simp = simp_l + simp_r;

        cout << "Step size : (left) : " << stepsize_l << " (right) : "<< stepsize_r << endl;
        cout << "Trapezoidal method : " << final_trap << endl;
        cout << "Simpsons method : " << final_simp << endl;
        return 0;
}
\end{lstlisting}

The results are :

\begin{center}
\begin{longtable}{|c|c|c|c|c|c|c|}
\hline
No. of bins & Step size & Step size & Trapezoidal & $\frac{Trapezoidal}{Theoretical}$ & Simpsons & $\frac{Simpsons}{Theoretical}$ \\
 & (left) & (right) & & & & \\ 
\hline
\hline
10 & 0.079370 & 0.062996 & 3.6313 & 1.0010 & 3.6277 & 1.0000 \\
\hline
100 & 0.007937 & 0.006300 & 3.62764 & 1.0000 & \cellcolor{codegreen}3.6276 & 1.0000 \\
\hline
1000 & 0.000794 & 0.000630 & \cellcolor{codegreen}3.6276 & 1.0000 & \cellcolor{codegreen}3.6276 & 1.0000 \\
\hline
10000 & 7.93700e-05 & 6.29960e-05 & \cellcolor{codegreen}3.6276 & 1.0000 & \cellcolor{codegreen}3.6276 & 1.0000 \\
\hline
\hline
\end{longtable}
\end{center}



\end{document}
