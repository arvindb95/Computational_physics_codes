\documentclass[a4paper,10pt]{article}
\usepackage[utf8]{inputenc}
\usepackage{amsmath}
\usepackage{xfrac}
\usepackage{float}
\usepackage[a4paper, total={7.5in, 10in}]{geometry}
\usepackage{longtable}
\usepackage[table]{xcolor}
\usepackage{listings}
\lstset{
breaklines=true
}
%http://tex.stackexchange.com/questions/116534/lstlisting-line-wrapping
\usepackage{hyperref}
\hypersetup{
    colorlinks=true,
    linkcolor=blue,
    citecolor=blue
}

\lstset{language=[77]Fortran,
  basicstyle=\ttfamily,
  keywordstyle=\color{red},
  commentstyle=\color{green},
  morecomment=[l]{!\ }% Comment only with space after !
}

\usepackage{color}
 
\definecolor{codegreen}{rgb}{0,0.6,0}
\definecolor{codegray}{rgb}{0.5,0.5,0.5}
\definecolor{codepurple}{rgb}{0.58,0,0.82}
\definecolor{backcolour}{rgb}{0.95,0.95,0.92}
 
\lstdefinestyle{mystyle}{
    backgroundcolor=\color{backcolour},   
    commentstyle=\color{codegreen},
    keywordstyle=\color{magenta},
    numberstyle=\tiny\color{codegray},
    stringstyle=\color{codepurple},
    basicstyle=\footnotesize,
    breakatwhitespace=false,         
    breaklines=true,                 
    captionpos=b,                    
    keepspaces=true,                 
    numbers=left,                    
    numbersep=5pt,                  
    showspaces=false,                
    showstringspaces=false,
    showtabs=false,                  
    tabsize=2
}
 
\lstset{style=mystyle}

%opening

\title{}
\author{}
\date{}

\begin{document}

\section*{Python code and analysis}
\begin{lstlisting}[language=python]
import numpy as np

## Comparing different algorithms to solve dy/dx = f(x,y)

def euler_method(function,x0,y0,x,stepsize):
    """
    function = dy/dx; x0,y0 are initial conditions; x is the point where y needs to be calculated.
    """
    x_array = np.arange(x0, x + stepsize, stepsize)
    y_array = np.zeros(len(x_array))
    y_array[0] = y0
    for i in range(1,len(x_array)):
        y_array[i] = y_array[i-1] + stepsize*function(x_array[i-1],y_array[i-1])

    return y_array[-1]

def partialx(function,x0,y0,stepsize):
    """
    Returns partial first derivative of the function "function" at the point x0,y0 with respect to x by considering points, one and two steps on either side of x0.
    """
    return (function(x0 - 2*stepsize, y0) - 8*function(x0 - stepsize, y0) + 8*function(x0 + stepsize, y0) - function(x0 + 2*stepsize, y0))/(12*stepsize)

def partialy(function,x0,y0,stepsize):
    """
    Returns partial first derivative of the function "function" at the point x0,y0 with respect to y by considering points, one and two steps on either side of y0.
    """
    return (function(x0, y0 - 2*stepsize) - 8*function(x0, y0 - stepsize) + 8*function(x0, y0 + stepsize) - function(x0, y0 + 2*stepsize))/(12*stepsize)

def taylor_method(function,x0,y0,x,stepsize):
    """
    function = dy/dx; x0,y0 are initial conditions; x is the point where y needs to be calculated.
    """
    x_array = np.arange(x0, x + stepsize, stepsize)
    y_array = np.zeros(len(x_array))
    y_array[0] = y0
    for i in range(1,len(x_array)):
        y_array[i] = y_array[i-1] + stepsize*function(x_array[i-1],y_array[i-1]) + (stepsize**2.0)*(partialx(function,x_array[i-1],y_array[i-1],10**(-7)) + function(x_array[i-1],y_array[i-1])*partialy(function,x_array[i-1],y_array[i-1],10**(-7)))

    return y_array[-1]

def adam_bashford_method(function,x0,y0,x,stepsize):
    """
    function = dy/dx; x0,y0 are initial conditions; x is the point where y needs to be calculated.
    """
    x_array = np.arange(x0, x + stepsize, stepsize)
    y_array = np.zeros(len(x_array))
    y_array[0] = y0
    y_array[1] = y0
    for i in range(2,len(x_array)):
        y_array[i] = y_array[i-1] + stepsize*(1.5*function(x_array[i-1],y_array[i-1]) - 0.5*function(x_array[i-2],y_array[i-2]))

    return y_array[-1]

def runge_kutta_method(function,x0,y0,x,stepsize):
    """
    function = dy/dx; x0,y0 are initial conditions; x is the point where y needs to be calculated.
    """
    x_array = np.arange(x0, x + stepsize, stepsize)
    y_array = np.zeros(len(x_array))
    y_array[0] = y0
    for i in range(1,len(x_array)):
        y_step = stepsize*function(x_array[i-1], y_array[i-1])
        y_array[i] = y_array[i-1] + stepsize*function(x_array[i-1] + (stepsize/2.0), y_array[i-1] + (y_step/2.0))

    return y_array[-1]
\end{lstlisting}

Results for y(1) and y(3) are given below ($\Delta$y is the difference from the analytical value):

\begin{center}
\begin{longtable}{|| c || c c || c c || c c || c c ||}
\hline
h & Euler y(1) & $\Delta$y(1) & Taylor y(1) & $\Delta$y(1) & Adam y(1) & $\Delta$y(1) & Runge y(1) & $\Delta$y(1) \\
 & & & & & Bashford & & Kutta & \\
\hline
0.5 & 0.75 & 0.1435 & 0.421875 & 0.1847 & 0.625 & 0.01847 & 0.587891 & 0.01864 \\
0.2 & 0.652861 & 0.04633 & 0.55319 & 0.05334 & 0.607383 & 0.0008524 & 0.604186 & 0.002345 \\
0.1 & 0.628157 & 0.02163 & 0.583102 & 0.02343 & 0.606581 & 5.053e-05 & 0.605987 & 0.0005433 \\
0.05 & 0.616984 & 0.01045 & 0.59562 & 0.01091 & 0.606522 & 9.105e-06 & 0.6064 & 0.0001309 \\
0.02 & 0.610629 & 0.004098 & 0.602359 & 0.004172 & 0.606527 & 3.604e-06 & 0.60651 & 2.05e-05 \\
0.01 & 0.608566 & 0.002035 & 0.604477 & 0.002054 & 0.60653 & 1.082e-06 & 0.606526 & 5.09e-06 \\
0.005 & 0.607545 & 0.001014 & 0.605512 & 0.001019 & 0.60653 & 2.932e-07 & 0.606529 & 1.268e-06 \\
0.002 & 0.606936 & 0.0004049 & 0.606125 & 0.0004056 & 0.606531 & 4.909e-08 & 0.60653 & 2.025e-07 \\
0.001 & 0.606733 & 0.0002023 & 0.606328 & 0.0002025 & 0.606531 & 1.245e-08 & 0.606531 & 5.058e-08 \\
\hline
\end{longtable}
\end{center}


\begin{center}
\begin{longtable}{|| c || c c || c c || c c || c c ||}
\hline
h & Euler y(3) & $\Delta$y(3) & Taylor y(3) & $\Delta$y(3) & Adam y(3) & $\Delta$y(3) & Runge y(3) & $\Delta$y(3) \\
 & & & & & Bashford & & Kutta & \\
\hline
0.5 & 0.0 & 0.01111 & 0.094551 & 0.08344 & 0.02124 & 0.01013 & 0.02999 & 0.01888 \\
0.2 & 0.00459 & 0.006519 & 0.026157 & 0.01505 & 0.012549 & 0.00144 & 0.012587 & 0.001478 \\
0.1 & 0.007791 & 0.003318 & 0.016062 & 0.004953 & 0.01148 & 0.0003708 & 0.011409 & 0.0002999 \\
0.05 & 0.009444 & 0.001665 & 0.01313 & 0.002021 & 0.011202 & 9.28e-05 & 0.011177 & 6.824e-05 \\
0.02 & 0.010443 & 0.0006664 & 0.011828 & 0.0007191 & 0.011124 & 1.479e-05 & 0.011119 & 1.035e-05 \\
0.01 & 0.010776 & 0.0003333 & 0.011455 & 0.0003461 & 0.011113 & 3.689e-06 & 0.011112 & 2.543e-06 \\
0.005 & 0.010942 & 0.0001666 & 0.011279 & 0.0001698 & 0.01111 & 9.212e-07 & 0.01111 & 6.302e-07 \\
0.002 & 0.011042 & 6.665e-05 & 0.011176 & 6.716e-05 & 0.011109 & 1.473e-07 & 0.011109 & 1.003e-07 \\
0.001 & 0.011076 & 3.333e-05 & 0.011142 & 3.345e-05 & 0.011109 & 3.681e-08 & 0.011109 & 2.504e-08 \\
\hline
\end{longtable}
\end{center}


\section*{Fortran90 code and analysis}
\begin{lstlisting}[language={[77]Fortran}]
function f(x,y) result(z)
    double precision x, y, z
    z = -x*y
end function f

function euler_method(x0,y0,x,stepsize) result(e)
    double precision f, x0, y0, x, stepsize, x1, y1, e
    integer n, i
    n = (x - x0)/stepsize
    do 10 i = 0, n
        y1 = y0 + stepsize*f(x0, y0)
        x1 = x0 + stepsize
        x0 = x1
        y0 = y1
    10 continue
    e = y1
end function euler_method

function partialx(x0,y0,stepsize) result(ddx)
    double precision f, x0, y0, stepsize, ddx
    ddx = (f(x0-2.0*stepsize,y0)-8.0*f(x0-stepsize,y0)+8.0*f(x0 + stepsize,y0)-f(x0+2.0*stepsize,y0))/(12.0*stepsize)
end function partialx

function partialy(x0,y0,stepsize) result(ddy)
    double precision f, x0, y0, stepsize, ddy
    ddy = (f(x0,y0-2.0*stepsize)-8.0*f(x0,y0-stepsize)+8.0*f(x0,y0+stepsize)-f(x0,y0+2.0*stepsize))/(12.0*stepsize)
end function partialy

function taylor_method(x0,y0,x,stepsize) result(t)
    double precision f, x0, y0, x, stepsize, x1, y1, t, partialx, partialy, h
    integer n, i
    h = 10E-7
    n = (x - x0)/stepsize
    do 20 i = 0, n
        y1 = y0 + stepsize*f(x0,y0) + (stepsize**2.0)*(partialx(x0,y0,h) + f(x0,y0)*partialy(x0,y0,h))
        x1 = x0 + stepsize
        x0 = x1
        y0 = y1
    20 continue
    t = y1
end function taylor_method

function adam_bashford_method(x0,y0,x,stepsize) result(a)
    double precision f, x0, y0, x, stepsize, x1, y1, x2, y2, a
    integer n, i
    n = (x - x0)/stepsize
    x1 = x0 + stepsize
    y1 = y0 + stepsize*f(x0,y0)
    do 30 i = 1 ,n
        y2 = y1 + stepsize*(1.5*f(x1,y1) - 0.5*f(x0,y0))
        x2 = x1 + stepsize
        x0 = x1
        x1 = x2
        y0 = y1
        y1 = y2
    30 continue
    a = y2
end function adam_bashford_method

function runge_kutta(x0,y0,x,stepsize) result(r)
    double precision f, x0, y0, x, x1, y1, stepsize, k, r
    integer n, i
    n = (x - x0)/stepsize
    do 40 i = 1,n
        k = stepsize*f(x0,y0)
        y1 = y0 + stepsize*f(x0 + stepsize*0.5, y0 + k*0.5)
        x1 = x0 + stepsize
        x0 = x1
        y0 = y1
    40 continue
    r = y1
end function runge_kutta

program solve_diff_eqn

implicit none
double precision x0,y0,x,stepsize
double precision euler_method, taylor_method, adam_bashford_method, runge_kutta

x0 = 0
y0 = 1
x = 1
stepsize = 0.001

print*,euler_method(x0, y0, x, stepsize)
x0 = 0
y0 = 1

print*,taylor_method(x0, y0, x, stepsize)

x0 = 0
y0 = 1

print*,adam_bashford_method(x0, y0, x, stepsize)

x0 = 0
y0 = 1

print*,runge_kutta(x0, y0, x, stepsize)

end program solve_diff_eqn

\end{lstlisting}
\newpage

Results for y(1) and y(3) are given below ($\Delta$y is the difference from the analytical value):

\begin{center}
\begin{longtable}{|| c || c c || c c || c c || c c ||}
\hline
h & Euler y(1) & $\Delta$y(1) & Taylor y(1) & $\Delta$y(1) & Adam y(1) & $\Delta$y(1) & Runge y(1) & $\Delta$y(1) \\
 & & & & & Bashford & & Kutta & \\
\hline
0.5 & 0.375 & 0.231531 & 0.210937 & 0.395593 & 0.28125 & 0.325281 & 0.587890 & 0.01864 \\
0.2 & 0.652861 & 0.04633 & 0.55319 & 0.05334 & 0.607383 & 0.0008524 & 0.724096 & 0.117565 \\
0.1 & 0.628157 & 0.02163 & 0.583102 & 0.02343 & 0.606581 & 5.053e-05 & 0.666451 & 5.99e-02 \\
0.05 & 0.616984 & 0.01045 & 0.59562 & 0.01091 & 0.606522 & 9.105e-06 & 0.636701 & 3.02e-02 \\
0.02 & 0.598416 & 8.114e-03 & 0.590311 & 1.621e-02 & 0.594394 & 1.213e-02 & 0.60651 & 2.05e-05 \\
0.01 & 0.602480 & 4.05e-03 & 0.598432 & 8.098e-03 & 0.600463 & 6.066e-03 & 0.606526 & 5.09e-06 \\
0.005 & 0.604507 & 2.023e-03 & 0.602484 & 4.046e-03 & 0.603497 & 3.032e-03 & 0.606529 & 1.268e-06 \\
0.002 & 0.606936 & 0.0004049 & 0.606125 & 0.0004056 & 0.606531 & 4.909e-08 & 0.607743 & 1.21e-03 \\
0.001 & 0.606733 & 0.0002023 & 0.606328 & 0.0002025 & 0.606531 & 1.245e-08 & 0.607137 & 6.06e-04 \\
\hline
\end{longtable}
\end{center}

\begin{center}
\begin{longtable}{|| c || c c || c c || c c || c c ||}
\hline
h & Euler y(3) & $\Delta$y(3) & Taylor y(3) & $\Delta$y(3) & Adam y(3) & $\Delta$y(3) & Runge y(3) & $\Delta$y(3) \\
 & & & & & Bashford & & Kutta & \\
\hline
0.5 & 0.0 & 0.01111 & 0.141826 &  0.130717 & 1.525e-03 & 9.583e-03 & 0.02999 & 0.01888 \\
0.2 & 0.00459 & 0.006519 & 0.026157 & 0.01505 & 0.012549 & 0.00144 & 2.161e-02 & 1.050e-02 \\
0.1 & 0.007791 & 0.003318 & 0.016062 & 0.004953 & 0.01148 & 0.0003708 & 1.525e-02 & 4.148e-03 \\
0.05 & 0.009444 & 0.001665 & 0.01313 & 0.002021 & 0.011202 & 9.28e-05 & 1.296e-02 & 1.854e-03 \\
0.02 & 9.816e-03 & 1.292e-03 & 1.115e-02 & 4.728e-05 & 1.047e-02 & 6.344e-04 & 0.011119 & 1.035e-05 \\
0.01 & 1.045e-02 & 6.565e-04 & 1.112e-02 & 1.159e-05 & 1.078e-02 & 3.251e-04 & 0.011112 & 2.545e-06 \\
0.005 & 1.077e-02 & 3.307e-04 & 1.111e-02 & 2.875e-06 & 1.094e-02 & 1.646e-04 & 0.01111 & 6.326e-07 \\
0.002 & 0.011042 & 6.665e-05 & 0.011176 & 6.716e-05 & 0.011109 & 1.473e-07 & 0.011175 & 6.692e-05 \\
0.001 & 0.011076 & 3.333e-05 & 0.011142 & 3.345e-05 & 0.011109 & 3.681e-08 & 0.011142 & 3.339e-05 \\
\hline
\end{longtable}
\end{center}

\end{document}
